\documentclass[11pt]{report}

\linespread{1.5}
\usepackage[a4paper, margin=2cm]{geometry}
\usepackage{subfiles}
\usepackage[hidelinks]{hyperref}
\usepackage{bookmark}
\usepackage{amssymb}
\usepackage{listings}
\usepackage{xcolor}
\usepackage{longtable}
\usepackage{mathtools}
\usepackage{graphicx}
\usepackage{rotating}
\usepackage{tikz}
\usepackage{caption}
\usepackage{subcaption}
\usepackage{graphicx}
\usepackage{array}
\usepackage{booktabs}
\usepackage{fancyhdr}
\graphicspath{{./}{./chapters/results/}{./chapters/generator/}{./chapters/theory/}}
\usepackage{float}
\usepackage[toc,page]{appendix}

% JSON prettification
\colorlet{punct}{red!60!black}
\definecolor{background}{HTML}{EEEEEE}
\definecolor{delim}{RGB}{20,105,176}
\colorlet{numb}{magenta!60!black}

\lstdefinelanguage{json}{
    basicstyle=\normalfont\ttfamily\small,
    numbers=left,
    numberstyle=\scriptsize,
    stepnumber=1,
    numbersep=8pt,
    showstringspaces=false,
    breaklines=true, 
    frame=lines,
    backgroundcolor=\color{background},
    literate=
     *{0}{{{\color{numb}0}}}{1}
      {1}{{{\color{numb}1}}}{1}
      {2}{{{\color{numb}2}}}{1}
      {3}{{{\color{numb}3}}}{1}
      {4}{{{\color{numb}4}}}{1}
      {5}{{{\color{numb}5}}}{1}
      {6}{{{\color{numb}6}}}{1}
      {7}{{{\color{numb}7}}}{1}
      {8}{{{\color{numb}8}}}{1}
      {9}{{{\color{numb}9}}}{1}
      {:}{{{\color{punct}{:}}}}{1}
      {,}{{{\color{punct}{,}}}}{1}
      {\{}{{{\color{delim}{\{}}}}{1}
      {\}}{{{\color{delim}{\}}}}}{1}
      {[}{{{\color{delim}{[}}}}{1}
      {]}{{{\color{delim}{]}}}}{1},
}

\lstdefinelanguage{python}{
    basicstyle=\normalfont\ttfamily\small,
    keywords = [1]{def, pass},
    keywordstyle=[2]\color{green},
    numbers=left,
    numberstyle=\scriptsize,
    stepnumber=1,
    numbersep=8pt,
    showstringspaces=false,
    breaklines=true, 
    frame=lines,
    backgroundcolor=\color{background},
    literate=
     *{0}{{{\color{numb}0}}}{1}
      {1}{{{\color{numb}1}}}{1}
      {2}{{{\color{numb}2}}}{1}
      {3}{{{\color{numb}3}}}{1}
      {4}{{{\color{numb}4}}}{1}
      {5}{{{\color{numb}5}}}{1}
      {6}{{{\color{numb}6}}}{1}
      {7}{{{\color{numb}7}}}{1}
      {8}{{{\color{numb}8}}}{1}
      {9}{{{\color{numb}9}}}{1}
      {:}{{{\color{punct}{:}}}}{1}
      {,}{{{\color{punct}{,}}}}{1}
      {\{}{{{\color{delim}{\{}}}}{1}
      {\}}{{{\color{delim}{\}}}}}{1}
      {[}{{{\color{delim}{[}}}}{1}
      {]}{{{\color{delim}{]}}}}{1},
}

\pagestyle{fancy}
\chead{Alexander Atack}
\lhead{}
\rhead{}

\begin{document}

\begin{titlepage}
    \begin{center}
        \vspace*{1cm}
        \Huge\textbf{Learning Latent Mappings to Satisfy Binary Constraints [DRAFT]}
        \vspace{1.5cm}
        \\\Large\textbf{Alexander Atack - 27745449}
        \\\Large{April 2019}
        \vspace{1.5cm}
        \\\Large{Supervised by Dr. David Toal}
        \\\Large{9983 words}
        \vspace{1.5cm}
        \\\small{This report is submitted in partial fulfillment of the requirements for the Aeronautics \& Astronautics MEng, Faculty of Engineering and the Environment, University of Southampton.}
        \vspace{1.5cm}
    \end{center}
    \small{
        I, Alexander Atack, declare that this thesis and the work presented in it are my own and has been generated by me as a result of my own original research.
        I confirm that:
        \begin{enumerate}
            \item this work has been done wholly or mainly while in candidature for a degree at this University;
            \item where any part of this thesis has previously been submitted for any other qualification at this University or any other institution, this has been clearly stated;
            \item where I have consulted the published work of others, this is always clearly attributed;
            \item where I have quoted from the work of others, the source is always given. With the exception of such quotations, this thesis is entirely my own work;
            \item I have acknowledged all main sources of help;
            \item where the thesis is based on work done by myself jointly with others, I have made clear exactly what was done by others and what I have contributed myself;
            \item none of this work has been published before submission.
        \end{enumerate}
    }
\end{titlepage}

\begin{abstract}
    Optimisation algorithms which are often used in engineering are not well suited to avoiding specific disallowed solutions such as those with degenerate geometries.
    This project introduces an adaptation of the generative adversarial network architecture in which a discriminator is trained to determine the viability of solutions, and a generator is trained to sample from a range of solutions that satisfy the discriminator.
    The architecture of the generator is also adapted so that its output can be parameterised by a constraint describing the space of viable solutions.
    In so doing, a latent space parameterised by a constraint vector is learned, allowing optimisation algorithms to explore only those solutions which will satisfy the constraint.

    Promising results are obtained, showing that the generator is capable of learning effective mappings that satisfy a range of constraints, including equality constraints.
    Some limitations of the proposed architecture are also explored, and the reasons behind them identified for improvement by future work.
\end{abstract}

\tableofcontents

\subfile{nomenclature/main}
\subfile{chapters/introduction/main}
\subfile{chapters/literature/main}
\subfile{chapters/theory/main}
\subfile{chapters/generator/main}
\subfile{chapters/discriminator/main}
\subfile{chapters/results/main}
\subfile{chapters/conclusion/main}
\subfile{bibliography/main}
\subfile{appendix/main}

\end{document}

\documentclass[../../main.tex]{subfiles}

\begin{document}

\chapter{Introduction} \label{chapter:introduction}

\section{Motivation} \label{section:motivation}

Many engineering challenges can be expressed as optimisation problems, in which the objective is to search a space of possibilities for a value which optimises some objective function.
In real world problems, the objective function is typically complicated enough that its maximum cannot be found analytically; in these cases, it is necessary to explore the solution space in order to find the optimal solution.
A number of algorithms exist to search this space efficiently, but often converge slowly when the error surface of the objective function changes abruptly.
This will often be the case when trying to optimise continuous properties, such as minimising the mass of a design, while simultaneously adhering to constraints, such as avoiding degenerate geometries.

Machine learning excels in exploiting underlying patterns in data to make predictions in the face of uncertainty.
This project explores the use of machine learning in constructing a differentiable mapping, parameterised by a constraint, between an arbitrary latent space and the space of solutions to a problem such that any point in the latent space, when mapped to the solution space, satisfies the constraint.

\section{Engineering problems} \label{section:engineeringProblems}

For the purposes of this project, an engineering problem is defined as an optimisation problem in which a solution $s$ is sought which optimises an arbitrary objective function $f(s)$, while also satisfying a constraint $c$.
Constraint satisfaction is determined by some function $h(c,s)$ which maps a constraint and solution to $\{\text{satisfied},\text{unsatisfied}\}$.
A solution which satisfies $h$ is referred to as viable.

The functions $f$ and $h$ are deterministic and belong to an engineering environment, which is thought of as an idealised mathematical model of a particular real-world problem.
While each environment has a unique constraint satisfaction function $h$, they may define a number of different continuous objective functions.

It is also assumed that both the solution and constraint are parameterised by a vector bounded by an hypercube in $m$ or $n$ dimensions respectively, denoted $S$ and $C$.
$h(c,s)$ is defined as long as $c\in C$ and $s\in S$.
\begin{equation}
    S=\{(s_1,...,s_m)\;|\;s_i\in\{a_S,b_S\},1\le i\le m\}
\end{equation}
\begin{equation}
    C=\{(c_1,...,c_n)\;|\;c_i\in\{a_C,b_C\},1\le i\le n\}
\end{equation}
where $a,\;b$ are the arbitrary bounds of the hypercube in each dimension.

Since both $S$ and $C$ are hypercubes, it is trivial to generate vectors which lie within them (Appendix \ref{appendix:samplingFromAxisAlignedHypercubes}).
As such, solving an engineering problem as defined here for a given constraint $c$ can be broken down into two steps: finding a solution $s$ for which $f(s)$ is suitably large; and ensuring that $s$ is viable.

\section{Learned latent mapping} \label{section:learnedLatentMapping}

The first step can already be achieved with conventional optimisation algorithms which have no knowledge of the objective surface.
When certain regions of the solution space do not satisfy the constraint, however, optimisation algorithms which work by exploring the solution space are unable to work efficiently since there is no natural way of reconciling the satisfaction of a constraint with a continuous objective function other than evaluating $h$ for every candidate solution.
If the regions satisfying the constraint are sparse, it may be a considerable amount of time before even a single viable solution is found, before then having to optimise its value according to the objective function.

One way around this dilemna is to utilise the fact that $h$ is constant within an environment to learn its underlying patterns using data from previous experiments.
This knowledge can be used to create a space of viable solutions $V_c$, specific to the constraint, such that any solution vector which is contained within $V_c$ satisfies $c$ by definition.
\begin{equation}
    V_c=\{s\;|\;s\in S,\;h(c,s)=\text{satisfied}\}
\end{equation}
While envisaging such a set is trivial, sampling from it is not.
One can imagine a generator function $g(c)$ which maps from an arbitrary latent space $L$ (from which samples can be easily drawn) to somewhere in $V_c$:
\begin{equation}
    g(l,c)\;:\;L,C\mapsto V_c
\end{equation}
\begin{equation}
    L=\{(l_1,...,l_k)\;|\;l_i\in\{a_L,b_L\},1\le i\le k\}
\end{equation}
$g$ can therefore be viewed as the inverse of $h$.
Once $V$ and $g$ have been learned, solving the engineering problem for a constraint is a matter of performing optimisation over $l\in L$ of $(f\circ g)\;(l,c)$ using any exploratory optimisation algorithm.
As such, finding a way of modelling $V_c$ and $g$ for any environment is the subject of this project.

\end{document}
